\chapter{Conclusion}\label{chap:conclusion}
In this thesis I have researched the topic of indoor SLAM application using a set of cheap single-point 
LIDAR sensors. I have used data collected from a single simulated flight to evaluate different sensor 
layouts and settings, by filtering this flight data to simulate a certain number of VL53L1X sensors and 
their parameters.

I have measured the performance of VL53L1X sensor using different operating modes. I have determined 
the maximum ranging distance, update rate and noise level of each operating mode. To investigate the 
effects of different operating modes on the SLAM performance in 2D, I have used 13 sensors in even 
layout on the horizontal plane and introduced each sensor parameter one by one.

I have found that by increasing the sensor resolution to 3x3 or 4x4, the SLAM performance is increased
and mapping works reliably. Sensor resolutions 1x1 and 2x2 have significantly higher update rate, but 
using these settings, the SLAM algorithm became unreliable.

As next I have investigated the effects of sampling rate and reduced it for resolutions 3x3 and 4x4 
accordingly. After proper tuning of Cartographer, the accuracy of the two resolutions proved to be similar,
while lower resolution provides more frequent pose estimates. Limiting the maximum distance of the two
setups had a significant effect of the performance and Cartographer couldn't track the drone's position
in wide open spaces anymore.

After simulating all parameters of the sensors, I have decreased the number of LIDARS while keeping an
even layout. The tracking proved to be accurate using at least 8 sensors in total.

The evaluation of 2D SLAM performance helped the process of 3D evaluation. In this case I have also 
chose an even layout and simulated 43 LIDARs in total and introduced each sensor parameter one by one.
3D mapping worked both for 3x3 and 4x4 sensor resolutions reliable. Even with lowered sampling rate
according to each resolution, the mapping quality proved to be adequate.
At last I have limited the maximum range to 2.75m in both cases. The ranging limit caused the SLAM 
algorithm to be unreliable and produced maps became messy.
The fact that 3D SLAM is unreliable even with 43 sensors, proves that VL53L1X sensors and Cartographer
SLAM cannot be used for 3D SLAM. Using higher number of sensors might increase the performance,
but placing 43 sensors on a real-world drone is already unfeasible.

At last I have tested the performance of 2D SLAM using 8 sensors, by mapping a different building 
that has rooms of varying size. The algorithm performed well in small to mid sized rooms, but in open
spaces it completely lost track and was not able to compensate this error. I have found that the system
works reliably, when walls on all sides are in range of all the sensors. Therefore the maximum room size
is 2.75x2.75m, where 2D SLAM using 8 VL53L1X sensors work reliably.







