\pagenumbering{roman}
\setcounter{page}{1}

\selecthungarian

%----------------------------------------------------------------------------
% Abstract in Hungarian
%----------------------------------------------------------------------------
\chapter*{Kivonat}\addcontentsline{toc}{chapter}{Kivonat}
Dolgozatomban bemutatom 
% Jelen dokumentum egy diplomaterv sablon, amely formai keretet ad 
% a BME Villamosmérnöki és Informatikai Karán végző hallgatók által 
% elkészítendő szakdolgozatnak és diplomatervnek. A sablon használata opcionális. 
% Ez a sablon \LaTeX~alapú, a \emph{TeXLive} \TeX-implementációval és a PDF-\LaTeX~fordítóval működőképes.


\vfill
\selectenglish


%----------------------------------------------------------------------------
% Abstract in English
%----------------------------------------------------------------------------
\chapter*{Abstract}\addcontentsline{toc}{chapter}{Abstract}
In this thesis I propose a system that uses single point LIDAR sensors to scan a multicopter's environment and 
build a 3D map of its surroundings using SLAM algorithm. The selected sensor is a VL53L1X time of flight LIDAR
sensor developed by STMicroelectronics. Due to recent COVID-19 pandemic, I had no access to the laboratory 
equipment. As a workaround I have used Gazebo simulator and simulated LIDAR measurements as close
real VL53L1X measurements as possible. 

Different camera and ranging sensor based solutions can be found for SLAM applications. Ranging sensor Based
setups mostly use heavy, sensitive but nonetheless highly accurate planar scanners. A solution for hobbyists 
and students, called Crazyflie Multi-ranger deck uses the 5 pieces of the same VL53L1X sensor 
for SLAM applications. Another product called Skydio2 is an autonomous drone that is capable of active 
tracking and object avoidance even in dense forests, using 6 fisheye cameras to scan its environment. This 
drone is the most reliable drone for autonomous tracking currently available.

I have used PX4 repository that uses Gazebo to simulate an Iris drone and placed two LIDAR sensors on it,
one on the top and one on the bottom. The two sensors make ranges of the surroundings evenly in a sphere 
shape. LIDAR and IMU measurements are recorded into a Rosbag file. 
Using a self-developed filtering script, I filter the recorded LIDAR measurements to simulate a number of 
VL53L1X sensors in predefined orientation and settings. The settings are the resolution, update rate and
maximum ranging distance. These parameters come from measurements done with the actual VL53L1X sensor.
Offline processing using the Rosbag filter allows to rapidly test sensor settings and different layouts. 
Because data of the same flight is used for all tests, their performance can be compared.





% I have done literature review on different camera and ranging sensor based SLAM solutions. I chose PX4 firmware
% package to simulate a quadcopter, that uses Gazebo with ROS wrapper. I have found that Cartographer SLAM is the 
% optimal choice for my use case, because it is capable of 3D SLAM, while other ROS packages only offer 2D 
% SLAM.

% Based on the literature review and the chosen main components, I have designed the necessary steps to
% build the planned system. I have selected the most relevant operating modes of VL53L1X LIDAR to be tested
% and designed a basic sensor layout for both 2D and 3D mapping. I have installed the necessary packages and
% created a working environment in Visual Studio Code to make development easier. The workspace contains 
% the PX4 firmware repository for simulations, Cartographer SLAM and necessary script, like a virtual remote
% controller. 





\vfill
\selectthesislanguage

\newcounter{romanPage}
\setcounter{romanPage}{\value{page}}
\stepcounter{romanPage}