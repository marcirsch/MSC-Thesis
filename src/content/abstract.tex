\pagenumbering{roman}
\setcounter{page}{1}

\selecthungarian

%----------------------------------------------------------------------------
% Abstract in Hungarian
%----------------------------------------------------------------------------
\chapter*{Kivonat}\addcontentsline{toc}{chapter}{Kivonat}
Dolgozatomban bemutatom TODO
% Jelen dokumentum egy diplomaterv sablon, amely formai keretet ad 
% a BME Villamosmérnöki és Informatikai Karán végző hallgatók által 
% elkészítendő szakdolgozatnak és diplomatervnek. A sablon használata opcionális. 
% Ez a sablon \LaTeX~alapú, a \emph{TeXLive} \TeX-implementációval és a PDF-\LaTeX~fordítóval működőképes.


\vfill
\selectenglish


%----------------------------------------------------------------------------
% Abstract in English
%----------------------------------------------------------------------------
\chapter*{Abstract}\addcontentsline{toc}{chapter}{Abstract}

I have reviewed basic aviation terminologies and similar products available on the market.
Crazyflie Multi-ranger deck uses the same VL53L1X sensor that I chose and use 5 pieces for a SLAM application.
Skydio2 is an autonomous drone that is capable of active tracking and object avoidance even in dense 
forests. This drone uses 6 fisheye cameras for autonomous flight and it seems to be the most reliable 
drone for autonomous tracking currently available.

I have done literature review on different camera and ranging sensor based SLAM solutions. I chose PX4 
firmware package to simulate a quadcopter, that uses Gazebo with ROS wrapper. I have found that Cartographer 
SLAM is the optimal choice for my use case, because it is capable of 3D SLAM, while other ROS packages only 
include 2D solutions.

Based on the literature review and the chosen main components, I have designed the necessary steps to
build the planned system. I have selected the most relevant operating modes of VL53L1X LIDAR to be tested
and designed a basic sensor layout for both 2D and 3D mapping. I have installed the necessary packages and
created a working environment in Visual Studio Code to make development easier. The workspace contains 
the PX4 firmware repository for simulations, Cartographer SLAM and necessary script, like a virtual remote
controller.

To be able to compare the performance of different SLAM algorithms, the same data needs to be used for All
setups. To do so, I have placed two LIDAR sensors on the simulated quadcopter to scan the whole sphere of 
the surroundings of the drone. ROS data of a flight is then collected into a rosbag, that contains the 
relevant measurements, that are high resolution LIDAR ranges and IMU data. The LIDAR measurements inside 
this rosbag is then filtered to simulate a certain number of VL53L1X sensors. Update rate, field of view 
and maximum ranging distance parameters are used to simulate a single sensor.




\vfill
\selectthesislanguage

\newcounter{romanPage}
\setcounter{romanPage}{\value{page}}
\stepcounter{romanPage}