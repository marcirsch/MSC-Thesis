%----------------------------------------------------------------------------
\chapter{\bevezetes}
%----------------------------------------------------------------------------


% A bevezető tartalmazza a diplomaterv-kiírás elemzését, történelmi előzményeit, a feladat indokoltságát
%  (a motiváció leírását), az eddigi megoldásokat, és ennek tükrében a hallgató megoldásának összefoglalását.

% A bevezető szokás szerint a diplomaterv felépítésével záródik, azaz annak rövid leírásával, hogy melyik 
% fejezet mivel foglalkozik.


A prerequisite for future expansion of mobile robots is the ability to navigate autonomously and safely in their ever-changing
environment without collision. The positioning of drones in ad hoc indoor environment where no infrastructure is available is 
needed to fulfill this prerequisite. To know the position of a robot in an unknown environment, the robot needs to build a map
if its surroundings so its position can be referenced. Mapping and positioning were considered as two separate problems 
at first and research focused on either of them, but the solution to these problems separately proved to be divergent. Only 
by considering these two problems together and finding solutions simultaneously, it became convergent 
\cite{durrant2006simultaneous}. This process is referred to as Simultaneous Localization And Mapping (SLAM).

The proposed issue by SLAM has many solutions depending on the use-case. SLAM technique can be used with a 
focus on improving localization accuracy in GPS degraded environments, by fusing the outputs of a SLAM algorithm with GPS and
IMU data \cite{hening20173d}. Or as presented in \cite{droeschel2018efficient}, the solution focuses on building highly accurate maps
online of building interiors or even outdoors, but as a byproduct relative position becomes more accurate as well. The 
above mentioned papers present a mapping approach that uses industrial-grade hexacopter and octocopter and the same 
precise LIDAR scanner that can measure up to 100m. 

Localization and mapping of the environment for a small quadcopters can be challenging, because adding any extra equipment means
extra weight that shortens flight time and changes flight behavior and the onboard computation capacity is usually limited. 
Skydio2 \cite{SkydioWebsite} uses 6 fisheye cameras and processes the data onboard using a GPGPU to track a moving target
while avoiding even the smallest branches in dense forests.
Rotating planar scanners are the other common choice, that are also used in the above mentioned papers, undoubtedly 
provide highly accurate scans and detailed maps. These LIDAR scanners are perfect for professional use and require reliable
octo- or hexacopters to carry. The popular Velodyne VLP-16 for example weighs 830g. Adding it to a quadcopter that is under 
1kg means significant extra weight and the drone might turn out to be uncapable of flying. 

By using stationary LIDAR sensors placed around the quadcopter, these unwanted effects can be eliminated and the weight of 
extra equipment can be kept significantly lower, than using a 3D LIDAR scanner discussed before. This solution might also
provide a cheap and computationally light alternative to camera or planar scanner based solutions. 
The selected single-point sensors need to be placed on a drone in a layout that offers an optimal scan of the
environment, therefore the movements of quadcopter during flight needs to be taken into account to be able to scan the
environment properly around the vehicle.

In this thesis the building and evaluation of a 3D positioning and mapping system is introduced, using Gazebo simulator. 
The system uses stationary LIDAR sensors placed on a simulated quadcopter to acquire distance measurements from its 
environment and uses SLAM technique to map its surrounding, estimate its position and produce an indoor map. 

In chapter \ref{chap:literature_review} basic aviation terminologies are introduced that will be used in the thesis, followed
by introduction of SLAM techniques and similar products on the market. Visual SLAM techniques describe the use of monocular 
cameras, stereo cameras or RGB-D cameras to be used for data acquisition, while ORB-SLAM and LSD-SLAM are two popular SLAM
techniques for processing data coming from these sensors. Ranging sensors are non-visual sensors that can use sound
or light to measure distance, like ultrasonic or LIDAR sensors. Two interesting researches are discussed, that use 
LIDAR scanners to improve position or map quality online. The algorithm developed by google called Cartographer SLAM 
is being introduced here. Cartographer is a is a software package that will be used for SLAM implementation and to evaluate 
layout design and settings of the stationary sensors. The algorithm will be used with the official ROS wrapper.
Lastly all necessary components of the simulation and their task is discussed. 
 
Chapter \ref{chap:system_design} gives an overview of VL53L1X LIDAR sensor produced by STMicroelectronics and describes the
relevant operating modes to be tested and the methodology of the measurement. These results will be used to accurately simulate
the sensors. In this chapter initial sensor layouts are proposed for both 2D and 3D SLAM operations.

To be able to compare the performance of different SLAM algorithms, I chose to use the same data for the evaluation of 
different setups and filter the LIDAR measurements to simulate different layouts. 
To do so, two LIDAR sensors are placed on the simulated quadcopter to scan the whole sphere of the 
surroundings of the drone. Data published to ROS during flight is collected into a rosbag, that contains the relevant 
measurements, that are high resolution LIDAR ranges and IMU data. The LIDAR measurements inside this rosbag is then 
filtered to simulate a certain number of VL53L1X sensors. Update rate, field of view and maximum ranging distance parameters
from the planned sensor measurements will be used to simulate a sensor. The filtered data is fed to Cartographer SLAM
that produces the map after proper tuning. 

Chapter \ref{chap:implementation} describes the results of the VL53L1X measurements that was planned in previous chapter.
VL53L1X LIDAR has an adjustable region of interest inside the field of view and by moving this region around the field of view,
higher resolutions can be achieved from 1x1 to 4x4. It is yet unknown how different parameters affect the SLAM performance, 
therefore these are introduced incrementally. 

For 2D SLAM evaluation the same layout is used in all steps, 13 sensors evenly distributed on the horizontal 
plane. At first the effects of different resolutions from 4x4 to 1x1 are evaluated, then 
the effects of lowered sampling rate and lastly the effects of maximum ranging distance. The performance of each setup
is determined by calculating RMS error between the estimated trajectory and the ground-truth trajectory, extracted 
from the unfiltered dataset.

The three introduced parameters are according to results of VL53L1X measurements described in the beginning of this chapter.
Using the optimal settings, the number of sensors is decreased until the lowest number if found, that can still provide a 
reliable SLAM performance. Instead of decreasing the number of sensors by one at each iteration, I chose
to do logarithmic search.

The above described evaluation methodology is used for 2D SLAM and the optimal LIDAR settings are used for the evaluation 
of 3D SLAM. Just like in case of 2D, the sensor parameters are introduced one by one to see their effects independently 
from each other. Lastly the number of sensors is decreased until the lowest number is found.
