%----------------------------------------------------------------------------
\chapter{\bevezetes}
%----------------------------------------------------------------------------


% A bevezető tartalmazza a diplomaterv-kiírás elemzését, történelmi előzményeit, a feladat indokoltságát
%  (a motiváció leírását), az eddigi megoldásokat, és ennek tükrében a hallgató megoldásának összefoglalását.

% A bevezető szokás szerint a diplomaterv felépítésével záródik, azaz annak rövid leírásával, hogy melyik fejezet mivel foglalkozik.


A prerequisite for future expansion of mobile robots is the ability to navigate autonomously and safely in their ever-changing
environment without collision. The positioning of drones in ad hoc indoor environment where no infrastructure is available is 
needed to fulfill this prerequisite. To know the position of a robot in an unknown environment, the robot needs to build a map
if its surroundings so its position can be referenced. Mapping and positioning were considered as two separate problems 
at first and research focused on either of them, but the solution to these problems separately proved to be divergent. Only 
by considering these two problems together and finding solutions simultaneously, it became convergent 
\cite{durrant2006simultaneous}. This process is referred to as Simultaneous Localization And Mapping (SLAM).

The proposed issue by SLAM has many solutions depending on the use-case. SLAM technique can be used with a 
focus on improving localization accuracy in GPS degraded environments, by fusing the outputs of a SLAM algorithm with GPS and
IMU data \cite{hening20173d}. Or as presented in \cite{droeschel2018efficient}, the solution focuses on building highly accurate maps
online of building interiors or even outdoors, but as a byproduct position becomes more accurate as well. The above mentioned 
papers present a mapping approach that uses industrial-grade hexacopter and octocopter and the same precise LIDAR scanner that can
measure up to 100m. 

Localization and mapping of the environment for a small quadcopters can be challenging, because adding any extra equipment means
extra weight that shortens flight time and changes flight behavior and the onboard computation capacity is limited. Rotating planar 
scanners, that are also used in the above mentioned papers, undoubtedly provide highly accurate scans and therefore highly detailed maps.
These LIDAR scanners are perfect for professional use and require reliable octo- or hexacopters to carry. 
The popular Velodyne VLP-16 for example weighs 830g. Adding it to a quadcopter that is under 1kg means significant extra weight and the 
drone might turn out to be uncapable of flying. 

By using stationary LIDAR sensors placed around the quadcopter, these unwanted effects can be eliminated and the weight of 
extra equipment can be kept significantly lower, than using a 3D LIDAR scanner discussed before. The sensors need to be placed
in a layout that offers an optimal scan of the environment, therefore the movements of quadcopter during flight needs to be 
taken into account to be able to scan the environment around the vehicle.

In this thesis the design of a 3D positioning and mapping system is being discussed, that uses stationary LIDAR sensors to 
acquire distance measurements from an Unmanned Aerial Vehicle's (UAV) surrounding and uses SLAM technique to map its 
surrounding, estimate its position and velocity. The goal is to provide a lightweight hardware and software alternative to 
accurate planar scanners, that are too heavy both computationally and physically for a quadcopter under 1kg. The focus during 
the design is on the quality of the map and accuracy of the position estimates.

In chapter \ref{chap:literature_review} basic aviation terminologies are introduced that will be used in the thesis, followed
by introduction of SLAM techniques and similar products on the market. Visual SLAM techniques describe the use of monocular 
cameras, stereo cameras or RGB-D cameras to be used for data acquisition, while ORB-SLAM and LSD-SLAM are two popular SLAM
techniques for processing data coming from these sensors. Ranging sensors are non-visual sensors that can use sound
or light to measure distance, like ultrasonic or LIDAR sensors. Two interesting researches are discussed, that use 
LIDAR scanners to improve position or map quality online. The algorithm developed by google called Cartographer SLAM 
is being introduced here. Cartographer is a is a software package that will be used for SLAM implementation and  to evaluate 
layout design and settings of the stationary sensors. The algorithm will be used with the official ROS wrapper.

In chapter \ref{chap:components} all necessary components and their task is discussed, that will be used in  
simulations. Simulations help to evaluate sensors, layout designs and SLAM algorithms without the need to build the system
physically. 
 
Chapter \ref{chap:system_design} gives an overview of VL53L1X LIDAR sensor produced by STMicroelectronics and describes the
relevant operating modes that will be measured. Initial sensor layouts are proposed for both 2D and 3D SLAM operations. 
Due to recent COVID-19 pandemic I haven't had access to laboratory equipment and had to cancel the building of the system on 
an actual quadcopter. All measurements will be conducted in Gazebo simulator.




This chapter describes the setup of the simulation environment, the process of data recording 

the chosen LIDAR sensor , first designs 
of the layout of stationary sensors and a unit for collection of measurements from the sensors.




TODO Complete introduction